% -*- TeX-master: "main"; fill-column: 72 -*-
\section{Package syntax and semantics}

In this section, we define the syntax and semantics of the \FooPackage
for \sbmlthreecore. We expound on the various data types and constructs
defined in this package, then in \sec{examples}, we provide complete
examples of using the constructs in example SBML models.

\subsection{Namespace URI and other declarations necessary for using
this package}
\label{xml-namespace}

Every SBML Level~3 package is identified uniquely by an XML namespace
URI. For an SBML document to be able to use a given SBML Level~3
package, it must declare the use of that package by referencing its URI.
The following is the namespace URI for this version of the \FooPackage
for SBML Level~3 Version~1:

\begin{center}
\uri{http://www.sbml.org/sbml/level3/version1/foo/version1}
\end{center}

In addition, SBML documents using a given package must indicate whether
understanding the package is required for complete mathematical
interpretation of a model, or whether the package is optional. This is
done using the attribute \token{required} on the \token{<sbml>} element
in the SBML document. For the \FooPackage the value of the required
attribute is \val{true}.

% \begin{figure}[ht!]
%   \centering
%   \includegraphics[width=0.9\textwidth]{figures/foo_version_1_complete.png}\\
% \caption{A UML representation of the \FooPackage. See
% \ref{conventions} for conventions related to this figure. }
%   \label{fig:foo_version_1_complete}
% \end{figure}

\subsection{Primitive data types}
\label{primitive-types}

Section~3.1 of the SBML Level~3 specification defines a number of
primitive data types and also uses a number of XML Schema 1.0 data types
\citep{biron:2000}. We assume and use some of them in the rest of this
specification, specifically \primtype{boolean}, \primtype{ID},
\primtype{SId}, \primtype{SIdRef}, and \primtype{string}. The \Foo
Package defines other primitive types; these are described below.

\TODO{check all necessary types from core are listed}

% ---------------------------------------------------------
\subsection{The extended \class{Model} class}
\label{extended-model-class}

% \begin{figure}[ht!]
%   \centering
%   \includegraphics[scale=0.6]{figures/foo_extended_model_uml.pdf}\\
% \caption{A UML representation of the extended \Model class for the
% \FooPackage. See \ref{conventions} for conventions related to this
% figure. }
%   \label{fig:foo_extended_model_uml}
% \end{figure}


\TODO{explain where Model comes from}

The \FooPackage extends the \class{Model} object with the addition of
a \ListOfApples object.

% ---------------------------------------------------------
\subsection{The \class{ListOfApples} class}
\label{listofapples-class}

\TODO{explain ListOfApples}

The \ListOfApples object derives from the \class{SBase} and inherits the
core attributes and subobjects from that class. It contains two or more
objects of type \Apple.

% ---------------------------------------------------------
\subsection{The \class{Apple} class}
\label{apple-class}

% \begin{figure}[ht!]
%   \centering
%   \includegraphics[scale=0.6]{figures/foo_apple_uml.pdf}\\
% \caption{A UML representation of the \Apple class for the
% \FooPackage. See \ref{conventions} for conventions related to this
% figure. }
%   \label{fig:foo_apple_uml}
% \end{figure}


\TODO{explain Apple}

The \Apple object derives from the \SBase class and thus inherits any
attributes and elements that are present on this class.
% ---------------------------------------------------------
\subsection{The extended \class{Compartment} class}
\label{extended-compartment-class}

% \begin{figure}[ht!]
%   \centering
%   \includegraphics[scale=0.6]{figures/foo_extended_compartment_uml.pdf}\\
% \caption{A UML representation of the extended \Compartment class for
% the \FooPackage. See \ref{conventions} for conventions related to
% this figure. }
%   \label{fig:foo_extended_compartment_uml}
% \end{figure}


\TODO{explain where Compartment comes from}

The \FooPackage extends the \class{Compartment} object with the addition
of
a \ListOfPears object.

% ---------------------------------------------------------
\subsection{The \class{ListOfPears} class}
\label{listofpears-class}

\TODO{explain ListOfPears}

The \ListOfPears object derives from the \class{SBase} and inherits the
core attributes and subobjects from that class. It contains zero or more
objects of type \Pear.

% ---------------------------------------------------------
\subsection{The \class{Pear} class}
\label{pear-class}

% \begin{figure}[ht!]
%   \centering
%   \includegraphics[scale=0.6]{figures/foo_pear_uml.pdf}\\
% \caption{A UML representation of the \Pear class for the \FooPackage.
% See \ref{conventions} for conventions related to this figure. }
%   \label{fig:foo_pear_uml}
% \end{figure}


\TODO{explain Pear}

The \Pear object derives from the \SBase class and thus inherits any
attributes and elements that are present on this class.
% ---------------------------------------------------------
\subsection{The extended \class{Species} class}
\label{extended-species-class}

% \begin{figure}[ht!]
%   \centering
%   \includegraphics[scale=0.6]{figures/foo_extended_species_uml.pdf}\\
% \caption{A UML representation of the extended \Species class for the
% \FooPackage. See \ref{conventions} for conventions related to this
% figure. }
%   \label{fig:foo_extended_species_uml}
% \end{figure}


\TODO{explain where Species comes from}

The \FooPackage extends the \class{Species} object with the addition of
a \ListOfBananas object.

% ---------------------------------------------------------
\subsection{The \class{ListOfBananas} class}
\label{listofbananas-class}

\TODO{explain ListOfBananas}

The \ListOfBananas object derives from the \class{SBase} and inherits
the core attributes and subobjects from that class. It contains one or
more objects of type \Banana.

% ---------------------------------------------------------
\subsection{The \class{Banana} class}
\label{banana-class}

% \begin{figure}[ht!]
%   \centering
%   \includegraphics[scale=0.6]{figures/foo_banana_uml.pdf}\\
% \caption{A UML representation of the \Banana class for the
% \FooPackage. See \ref{conventions} for conventions related to this
% figure. }
%   \label{fig:foo_banana_uml}
% \end{figure}


\TODO{explain Banana}

The \Banana object derives from the \SBase class and thus inherits any
attributes and elements that are present on this class.
% ---------------------------------------------------------
\subsection{The \class{Bowl} class}
\label{bowl-class}

% \begin{figure}[ht!]
%   \centering
%   \includegraphics[scale=0.6]{figures/foo_bowl_uml.pdf}\\
% \caption{A UML representation of the \Bowl class for the \FooPackage.
% See \ref{conventions} for conventions related to this figure. }
%   \label{fig:foo_bowl_uml}
% \end{figure}


\TODO{explain Bowl}

The \Bowl object derives from the \SBase class and thus inherits any
attributes and elements that are present on this class.
A \Bowl contains exactly one \ListOfPears element.
% ---------------------------------------------------------
\subsection{The \class{Plate} class}
\label{plate-class}

% \begin{figure}[ht!]
%   \centering
%   \includegraphics[scale=0.6]{figures/foo_plate_uml.pdf}\\
% \caption{A UML representation of the \Plate class for the
% \FooPackage. See \ref{conventions} for conventions related to this
% figure. }
%   \label{fig:foo_plate_uml}
% \end{figure}


\TODO{explain Plate}

The \Plate object derives from the \SBase class and thus inherits any
attributes and elements that are present on this class.
A \Plate contains exactly one \ListOfApples element.
A \Plate contains at most one \ListOfBananas element.
