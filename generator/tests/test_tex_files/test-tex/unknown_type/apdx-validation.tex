% -*- TeX-master: "main"; fill-column: 72 -*-

\section{Validation of SBML documents}
\label{apdx-validation}

\subsection{Validation and consistency rules}
\label{validation-rules}

This section summarizes all the conditions that must (or in some cases,
at least \emph{should}) be true of an SBML Level~3 Version~1 model that
uses the \testAttributesPackage. We use the same conventions as are used
in the SBML Level~3 Version~1 Core specification document. In
particular, there are different degrees of rule strictness. Formally,
the differences are expressed in the statement of a rule: either a rule
states that a condition \emph{must} be true, or a rule states that it
\emph{should} be true. Rules of the former kind are strict SBML
validation rules---a model encoded in SBML must conform to all of them
in order to be considered valid. Rules of the latter kind are
consistency rules. To help highlight these differences, we use the
following three symbols next to the rule numbers:

\begin{description}

\item[\hspace*{6.5pt}\vSymbol\vsp] A \vSymbolName indicates a
\emph{requirement} for SBML conformance. If a model does not follow this
rule, it does not conform to the \testAttributesPackage specification.
(Mnemonic intention behind the choice of symbol: ``This must be
checked.'')

\item[\hspace*{6.5pt}\cSymbol\csp] A \cSymbolName indicates a
\emph{recommendation} for model consistency. If a model does not follow
this rule, it is not considered strictly invalid as far as the
\testAttributesPackage specification is concerned; however, it indicates
that the model contains a physical or conceptual inconsistency.
(Mnemonic intention behind the choice of symbol: ``This is a cause for
warning.'')

\item[\hspace*{6.5pt}\mSymbol\msp] A \mSymbolName indicates a strong
recommendation for good modeling practice. This rule is not strictly a
matter of SBML encoding, but the recommendation comes from logical
reasoning. As in the previous case, if a model does not follow this
rule, it is not strictly considered an invalid SBML encoding. (Mnemonic
intention behind the choice of symbol: ``You're a star if you heed
this.'')

\end{description}

The validation rules listed in the following subsections are all stated
or implied in the rest of this specification document. They are
enumerated here for convenience. Unless explicitly stated, all
validation rules concern objects and attributes specifically defined in
the \testAttributesPackage package.

For \notice convenience and brevity, we use the shorthand
``\token{test:\-x}'' to stand for an attribute or element name \token{x}
in the namespace for the \testAttributesPackage package, using the
namespace prefix \token{test}. In reality, the prefix string may be
different from the literal ``\token{test}'' used here (and indeed, it
can be any valid XML namespace prefix that the modeler or software
chooses). We use ``\token{test:\-x}'' because it is shorter than to
write a full explanation everywhere we refer to an attribute or element
in the \testAttributesPackage namespace.

\subsubsection*{General rules about this package}

\validRule{test-10101}{To conform to the \testAttributesPackage
specification for SBML Level~3 Version~1, an SBML document must declare
\uri{http://www.sbml.org/sbml/level3/version1/test/version1} as the
XMLNamespace to use for elements of this package. (Reference: SBML
Level~3 Specification for testAttributes, Version~1
\sec{xml-namespace}.)}

\validRule{test-10102}{Wherever they appear in an SBML document,
elements and attributes from the \testAttributesPackage must use the
\uri{http://www.sbml.org/sbml/level3/version1/test/version1} namespace,
declaring so either explicitly or implicitly. (Reference: SBML Level~3
Specification for testAttributes, Version~1 \sec{xml-namespace}.)}

\subsubsection*{General rules about identifiers}

\validRule{test-10301}{(Extends validation rule \#10301 in the
\sbmlthreecore specification. TO DO list scope of ids) (Reference: SBML
Level~3 Version~1 Core, Section~3.1.7.)}

\validRule{test-10302}{The value of a \token{test:\-id} must conform to
the syntax of the \class{SBML} data type \primtype{SId} (Reference: SBML
Level~3 Version~1 Core, Section~3.1.7.)}

\TODO{ANY LIST OF ELEMENTS THAT HAVE ATTRIBUTES}

\subsubsection*{Rules for the extended \class{SBML} class}

\validRule{test-20101}{In all SBML documents using the
\testAttributesPackage, the \class{SBML} object must have the
\token{test:\-required} attribute. (Reference: SBML Level~3 Version~1
Core, Section~4.1.2.)}

\validRule{test-20102}{The value of attribute \token{test:\-required} on
the \class{SBML} object must be of data type \primtype{boolean}.
(Reference: SBML Level~3 Version~1 Core, Section~4.1.2.)}

\validRule{test-20103}{The value of attribute \token{test:\-required} on
the \class{SBML} object must be set to \val{false}. (Reference: SBML
Level~3 Specification for testAttributes, Version~1
\sec{xml-namespace}.)}

\subsubsection*{Rules for \class{UnknownType} object}

\validRule{test-20201}{An \UnknownType object may have the optional SBML
Level~3 Core attributes \token{metaid} and \token{sboTerm}. No other
attributes from the SBML Level~3 Core namespaces are permitted on an
\UnknownType. (Reference: SBML Level~3 Version~1 Core, Section~3.2.)}

\validRule{test-20202}{An \UnknownType object may have the optional SBML
Level~3 Core subobjects for notes and annotations. No other elements
from the SBML Level~3 Core namespaces are permitted on an \UnknownType.
(Reference: SBML Level~3 Version~1 Core, Section~3.2.)}

\validRule{test-20203}{An \UnknownType object may have the optional
attribute \token{test:\-attribue}. No other attributes from the SBML
Level~3 testAttributes namespaces are permitted on an \UnknownType
object. (Reference: SBML Level~3 Specification for testAttributes,
Version~1, \sec{unknowntype-class}.)}

\validRule{test-20204}{FIXME: Encountered an unknown attribute type fred
in ValidationRulesForClass (Reference: SBML Level~3 Specification for
testAttributes, Version~1, \sec{unknowntype-class}.)}


